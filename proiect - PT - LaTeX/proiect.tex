% preamble

\documentclass{article}
%% \usepackage{times}
\usepackage{latexsym}
\usepackage{url}
\usepackage{hyperref}
\hypersetup{colorlinks=true}
\usepackage{graphicx}

\begin{document}
	
	% top matter
	
	\title{Programming Techniques Project, Spring 2016 Application 3: Interpreter for mathematical expressions}
	\author{Abu-Ras Mohamed Ata Radu and Calina Constantin Alin }
	\date{Computer Science English Section First Year Group 10105A}
	\maketitle
	\pagebreak
	
\section{Problem Statement}
Write an interpreter for a simple language that recognizes both neg-ative and positive integer constants, variable assignment and the basic mathematical operators: + , $ - $ , $ * $, / and ∧ for exponentiation. Once read the expressionsshould be represented as binary trees which will be used to compute the result.The interpreter should present the user with a Read Eval Print Loop (REPL).
\linebreak
\linebreak
Here is the explanation of how the imput can be given:
\begin{enumerate}
\item  You may enter any variable name followed by "=" and an integer to declare it
\item  You may reinitialize a variable whenever you want 
\item  You may simply enter a mathematical expression using integers and basic operators
\item  You may enter a mathematical expression using both integers and the variables previously declared
\end{enumerate}
An example session in the interpreter would look like this:\\

$>$ a = 6\\

- 6\\

$>$ b = 2\\

- 2\\

$>$ aˆ2 + 2*b + 2\\

- 42\\

\section{Pseudocode}

\section{Application design}
\subsection {The high level architectural overview of the application:}
\begin{itemize}
\item The application uses 4 main functions to resolve the given task and a number of other auxiliary functions that are called by the main ones in order to reduce the ammount of written code and make them more organised and structed. Such auxiliary functions are :  declarevar , push , priority , isoperator . 
\item The 4 main functions mentioned above are the following : infix2postfix , maketree , evaluatetree , read :
	  \begin{itemize}
		 \item infix2postfix : this function takes the mathematical expression as given and translates it into a postfix form which is needed later . This form is also known as the polish form .
		 \item maketree : the function that creates a binary tree using the polish form created by infix2postfix . The tree's structure is based on a few characteristics :
									 \begin{itemize}
									   \item wheter the value of a node in the tree is a mathematical operator or a variable(or simply a given integer)
									   \item the priority of the mathematical operators one to another
									 \end{itemize}
		\item evaluatetree : this function is used to compute the expression and simply returns an integer 
		\item read : this is the function that uses the other ones and resolves the following tasks : reads the expression , removes excess spaces , transforms the expesion into a postfix order , makes a binary tree based off that order and then evaluates the expression .
	\end{itemize}
\end{itemize}

\subsection {The specification of the input:}
The input may have one of the following 2 forms :
	\begin{itemize}
		\item a declaration of variables which may contain one variable and one value per line
		\item a mathematical expression using the basic mathematical operators stated in the hypothesis and either an integer or a variable that has been previously declared
	\end{itemize}
	
\subsection {The specification of the output:}
The output , depending on the form of the input , may be :
	\begin{itemize}
		\item in case of a variable declaration the output is the value of the declared variable
		\item in case of a mathematical expression the output will be an integer which results after computing the given expression 
	\end{itemize}
\subsection {The list of all the modules in the application and their description:}
\begin{itemize} 
	\item list : contains functions that help to create and check different states of a list
	\item polish : is used to transform a given expression to its polish form 
	\item tree : has functions that create a binary tree from a polish form of an expression and evaluates it
	\item inOut : in this module there are a few auxiliary functions used to read and print a variable or integer as well as the main functions which uses all the other modules to resolve the given task
\end{itemize}

\subsection {The list of all the functions in the application, grouped by modules:}
\begin{itemize} 
	\item list :
	\begin{itemize}
		\item int isempty : a function that , given a stack , returns wheter it's emtpy or not
		\item void emptystack : a function that , given a stack , sets its top to -1
		\item void push : a function that adds a given item to a given stack
		\item struct node* pop : a function that , given a stack , saves the data stored in the top node , deletes it from the stack and then returns the value
	\end{itemize}
	\item polish :
	\begin{itemize}
		\item int isoperator : a function that check if a given char represents a basic mathematical operator
		\item int priority : a function that calculates the priority of a given operator and returns 1 if the operator is $+$ , $-$ or 2 if the operator is $/$ , $*$ , \^\ 
		\item void infix2postfix : this function has 3 parameeters : an infix form of an expression ( the default one), an array where the postfix form should be saved (the polish form) and a bool which tells wheter there's a space in the infix form . It goes throw the infix form and depending on the state of the read char (operator or digit) it creates the postfix which will be later used
	\end{itemize}
	\item tree :
	\begin{itemize}
		\item void maketree : a function that given a postfix form of an expression and the root creates a binary tree in which it stores integers and operators in a way that every 2 integers should have an operator as a common father
		\item long long evaluatetree : a funtion that , given a node , checks if it is an operator firstly and then performs the operation defined by the operator between the sons of the node . If the given node is not an operator then the function returns the number stored in the node
	\end{itemize}
	\item inOut : 
	\begin{itemize}
		\item void display : a print function
		\item void declare\_var : a function that , given a line in which a declaration is written (contains a $=$) , it stores the value of the declared variable in an array
		\item void read : a function that , given an expression , check wheter it is a variable declaration or a mathematical expression . If it is a variable declaration then it stores the value as mentioned above and prints the value on the next line . If it is a mathematical expression then it performs the following steps:
		\begin{itemize}
			\item changes its order from infix to postfix
			\item creates a binary tree based off the postfix
			\item prints an integer using the function "evaluatetree" which represents the result of the mathematical expression
		\end{itemize}
	\end{itemize}
\end{itemize}
\section{Conclusions}

\section{References}



	
\end{document}